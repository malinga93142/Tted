% Compile with: lualatex documentation.tex
\documentclass[11pt,a4paper]{article}
\usepackage{fontspec}
\usepackage{geometry}
\usepackage{xcolor}
\usepackage{listings}
\usepackage{tcolorbox}
\usepackage{hyperref}
\usepackage{fancyhdr}
\usepackage{titlesec}
\setmainfont{Linux Libertine O}
\geometry{margin=1in}

% Color definitions
\definecolor{codebackground}{RGB}{248,248,248}
\definecolor{commandcolor}{RGB}{0,102,204}
\definecolor{commentcolor}{RGB}{102,102,102}
\definecolor{titlecolor}{RGB}{0,51,102}

% Code listing style
\lstset{
  basicstyle=\ttfamily\small,
  backgroundcolor=\color{codebackground},
  frame=single,
  rulecolor=\color{gray!30},
  numbers=none,
  breaklines=true,
  postbreak=\mbox{\textcolor{red}{$\hookrightarrow$}\space},
  tabsize=2,
  showstringspaces=false,
  captionpos=b
}

% Title formatting
\titleformat{\section}
  {\Large\bfseries\color{titlecolor}}
  {\thesection}{1em}{}

\titleformat{\subsection}
  {\large\bfseries\color{titlecolor}}
  {\thesubsection}{1em}{}

% Header and footer
\pagestyle{fancy}
\fancyhf{}
\fancyhead[L]{\textit{File Editor Tool}}
\fancyhead[R]{\textit{User Manual}}
\fancyfoot[C]{\thepage}

% Command box environment
\newtcolorbox{commandbox}[1]{
  colback=codebackground,
  colframe=commandcolor,
  fonttitle=\bfseries,
  title=#1,
  arc=2mm
}

\title{\Huge\bfseries File Editor Tool(Tted)\\\large Command-Line Text File Manipulation}
\author{Version 1.0\\Malinga RK}
\date{\today}

\begin{document}

\maketitle
\tableofcontents
\newpage

\section{Introduction}

The File Editor Tool is a command-line utility written in C for viewing and manipulating text files. It provides two distinct operating modes: \textbf{Terminal Mode} (preview only) and \textbf{In-Place Edit Mode} (file modification).

\subsection{Key Features}

\begin{itemize}
  \item View specific lines or line ranges
  \item Delete lines from files
  \item Append text after specific lines
  \item Replace text throughout files
  \item Safe preview mode (default)
  \item In-place editing with \texttt{-i} flag
\end{itemize}

\section{Operating Modes}

\subsection{Terminal Mode (Default)}

Terminal mode is the \textbf{safe default} operating mode. It displays the result of operations without modifying the original file. This allows you to preview changes before committing them.

\begin{tcolorbox}[colback=yellow!10,colframe=orange!80,title=Safety Feature]
All modification commands (\texttt{--delete}, \texttt{--append}, \texttt{--replace}) run in preview mode by default. The file is never modified unless the \texttt{-i} flag is explicitly provided.
\end{tcolorbox}

\subsection{In-Place Edit Mode}

To actually modify files, prefix your command with the \texttt{-i} flag. This mode creates a temporary file, performs the operation, and replaces the original file.

\section{Command Reference}

\subsection{Reading Operations}

These operations always work in terminal mode (read-only).

\subsubsection{View Single Line}

\begin{commandbox}{Syntax}
\begin{lstlisting}
./program --line [line_number] [file]
\end{lstlisting}
\end{commandbox}

\textbf{Description:} Displays a specific line from the file.

\textbf{Example:}
\begin{lstlisting}
./program --line 5 myfile.txt
\end{lstlisting}

\subsubsection{View Line Range}

\begin{commandbox}{Syntax}
\begin{lstlisting}
./program --lines [start,end] [file]
\end{lstlisting}
\end{commandbox}

\textbf{Description:} Displays a range of lines with line numbers.

\textbf{Example:}
\begin{lstlisting}
./program --lines 10,20 myfile.txt
\end{lstlisting}

Output format:
\begin{lstlisting}
 10:This is line 10
 11:This is line 11
 12:This is line 12
 ...
\end{lstlisting}

\subsection{Modification Operations}

\subsubsection{Delete Lines}

\begin{commandbox}{Terminal Mode (Preview)}
\begin{lstlisting}
./program --delete [start,end] [file]
\end{lstlisting}
\end{commandbox}

\begin{commandbox}{In-Place Mode (Modifies File)}
\begin{lstlisting}
./program -i --delete [start,end] [file]
\end{lstlisting}
\end{commandbox}

\textbf{Description:} Removes the specified line range from the file.

\textbf{Examples:}
\begin{lstlisting}
# Preview deletion
./program --delete 5,10 document.txt

# Actually delete lines 5-10
./program -i --delete 5,10 document.txt
\end{lstlisting}

\subsubsection{Append Text}

\begin{commandbox}{Terminal Mode (Preview)}
\begin{lstlisting}
./program --append [line_number] [file] "text"
\end{lstlisting}
\end{commandbox}

\begin{commandbox}{In-Place Mode (Modifies File)}
\begin{lstlisting}
./program -i --append [line_number] [file] "text"
\end{lstlisting}
\end{commandbox}

\textbf{Description:} Inserts text as a new line after the specified line number.

\textbf{Examples:}
\begin{lstlisting}
# Preview append
./program --append 3 notes.txt "This is a new line"

# Actually append
./program -i --append 3 notes.txt "This is a new line"
\end{lstlisting}

\subsubsection{Replace Text}

\begin{commandbox}{Terminal Mode (Preview)}
\begin{lstlisting}
./program --replace "old_text/new_text" [file]
\end{lstlisting}
\end{commandbox}

\begin{commandbox}{In-Place Mode (Modifies File)}
\begin{lstlisting}
./program -i --replace "old_text/new_text" [file]
\end{lstlisting}
\end{commandbox}

\textbf{Description:} Replaces the first occurrence of \texttt{old\_text} with \texttt{new\_text} on each line.

\textbf{Examples:}
\begin{lstlisting}
# Preview replacement
./program --replace "hello/goodbye" letter.txt

# Actually replace
./program -i --replace "hello/goodbye" letter.txt
\end{lstlisting}

\section{Usage Examples}

\subsection{Workflow Example}

A typical workflow involves previewing changes before applying them:

\begin{lstlisting}
# 1. View the current state
./program --lines 1,50 config.txt

# 2. Preview a deletion
./program --delete 10,15 config.txt

# 3. If satisfied, apply the change
./program -i --delete 10,15 config.txt

# 4. Verify the result
./program --lines 1,50 config.txt
\end{lstlisting}

\subsection{Batch Operations}

\begin{lstlisting}
# Preview multiple changes
./program --delete 5,10 data.txt
./program --append 20 data.txt "New section header"
./program --replace "old_version/new_version" data.txt

# Apply all changes
./program -i --delete 5,10 data.txt
./program -i --append 20 data.txt "New section header"
./program -i --replace "old_version/new_version" data.txt
\end{lstlisting}

\section{Error Handling}

\subsection{Common Errors}

\begin{description}
  \item[Invalid line number] The specified line number exceeds the total lines in the file.
  \item[Invalid line range] The start line is greater than the end line, or either exceeds the file length.
  \item[Cannot open file] The specified file does not exist or lacks read permissions.
  \item[Cannot create temporary file] Insufficient permissions or disk space for in-place editing.
\end{description}

\subsection{Error Messages}

\begin{lstlisting}
invalid line number
invalid line range
Invalid range: Operation is not permitted
Error: Cannot open file
Error: Cannot create temporary file
Error: Text to append is required
Error: Replace format should be "OldText/NewText"
\end{lstlisting}

\section{Technical Details}

\subsection{Line Numbering}

\begin{itemize}
  \item Line numbers start at 1 (not 0)
  \item Line ranges are inclusive: \texttt{5,10} includes both lines 5 and 10
  \item Maximum line number supported: 999,999,999 (limited by \texttt{unsigned int})
\end{itemize}

\subsection{Buffer Limitations}

\begin{itemize}
  \item Maximum line length: 512 characters (including newline)
  \item Lines exceeding this limit will be processed in chunks
  \item Maximum line number length in commands: 8 digits
\end{itemize}

\subsection{File Operations}

\textbf{In-Place Mode Process:}
\begin{enumerate}
  \item Read from original file
  \item Write modified content to \texttt{temp\_file.tmp}
  \item Close both files
  \item Remove original file
  \item Rename temporary file to original filename
\end{enumerate}

\section{Best Practices}

\subsection{Safety Guidelines}

\begin{itemize}
  \item Always preview changes with terminal mode first
  \item Keep backups of important files before using \texttt{-i} mode
  \item Test operations on sample files before production use
  \item Use version control systems for critical files
\end{itemize}

\subsection{Performance Tips}

\begin{itemize}
  \item For large files, specify exact line ranges rather than reading entire files
  \item Batch multiple preview operations before applying changes
  \item Consider file size when performing replace operations
\end{itemize}

\section{Compilation}

To compile the program:

\begin{lstlisting}[language=bash]
gcc -o filedit rtt.c -Wall -Wextra
\end{lstlisting}

Recommended compilation flags:
\begin{itemize}
  \item \texttt{-Wall}: Enable all warnings
  \item \texttt{-Wextra}: Enable extra warnings
  \item \texttt{-O2}: Optimization level 2 (optional)
  \item \texttt{-std=c99}: Use C99 standard (optional)
\end{itemize}

\section{Quick Reference}

\begin{table}[h]
\centering
\begin{tabular}{|l|l|l|}
\hline
\textbf{Command} & \textbf{Mode} & \textbf{Effect} \\
\hline
\texttt{--line N file} & Terminal & View line N \\
\texttt{--lines N,M file} & Terminal & View lines N-M \\
\texttt{--delete N,M file} & Terminal & Preview deletion \\
\texttt{-i --delete N,M file} & In-place & Delete lines \\
\texttt{--append N file "text"} & Terminal & Preview append \\
\texttt{-i --append N file "text"} & In-place & Append text \\
\texttt{--replace "a/b" file} & Terminal & Preview replace \\
\texttt{-i --replace "a/b" file} & In-place & Replace text \\
\hline
\end{tabular}
\caption{Command Quick Reference}
\end{table}

\end{document}
